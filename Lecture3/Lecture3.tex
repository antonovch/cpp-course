\documentclass[12pt]{article}
\usepackage{../common/cpp-lectures}

\title{Лекція 3. Основи C++.}

\begin{document}
    %\begin{titlepage}
	\maketitle
	%\end{titlepage}
    \abstract{Ця лекція знайомить з основими принципами, особливостями і правилами мови C++.}

	%\newpage
	\tableofcontents
	%\newpage
	
	\section{Загальні відомості про C++}
	Переконайтесь, що ви розумієте наступні терміни та концепції
	\begin{enumerate}
		\item C++ визначає низку зарезервованих ключових слів для свого використання, які не можливо використовувати для назв своїх сутностей. Повний список можна знайти \href{https://en.cppreference.com/w/cpp/keyword}{тут}.
		\item C++ складається з декількох сутностей. Сутності програми C++ -- це значення (values), об'єкти (objects), посилання (references), функції (functions), перечислювачі (enumerators), типи (types), члени класу (class members), шаблони (templates), спеціалізації шаблонів (template specializations), простори імен (namespaces). Макроси препроцесора не є об'єктами C++.
		\item \href{https://en.cppreference.com/w/cpp/language/declarations}{Декларації}, що задають повний тип сутності, яка буде використовуватись далі в програмі, і \href{https://en.cppreference.com/w/cpp/language/definition}{дефініції}, які задають сутність повністю і необхідні для її безпосереднього використання.
		\item \href{https://en.cppreference.com/w/cpp/language/types}{Базові типи}.
		\item \href{https://en.cppreference.com/w/cpp/language/identifiers}{Назви сутностей}.
		\item \href{https://en.cppreference.com/w/cpp/language/expressions}{Вирази}.
		\item \href{https://en.cppreference.com/w/cpp/language/if}{Оператор if}.
		\item \href{https://en.cppreference.com/w/cpp/language/switch}{Оператор switch}.
		\item Цикли \href{https://en.cppreference.com/w/cpp/language/for}{for} та \href{https://en.cppreference.com/w/cpp/language/range-for}{range-for}.
		\item Цикл \href{https://en.cppreference.com/w/cpp/language/while}{while}.
		\item Ключові слова \href{https://en.cppreference.com/w/cpp/language/continue}{continue} та \href{https://en.cppreference.com/w/cpp/language/break}{break}.
		\item \href{https://en.cppreference.com/w/cpp/language/scope}{Область існування}.
		\item \href{https://en.cppreference.com/w/cpp/language/cv}{Константність}.
		\item \href{https://en.cppreference.com/w/cpp/language/reference}{Посилання}.
		\item \href{https://en.cppreference.com/w/cpp/language/pointer}{Показники}.
	\end{enumerate}

	\sec{Завдання}
	Ознайомитись зі списком вище, а також з іншими розділами, які вас можуть зацікавити зі \href{https://en.cppreference.com/w/cpp/language}{списку}, використовуючи надані посилання, або літературу.

	\sec{Література}
	
	\begin{itemize}
		\item Stroustrup, Bjarne. The C++ programming language. Pearson Education, 2013.
		\item Lippman, S., Lajoie, J., \& Moo, B. C++ Primer, 5th Edition. Addison-Wesley Professional, 2012.
		\item Meyers, Scott. Effective C++: 55 Specific Ways to Improve Your Programs and Designs (3rd ed.). O'Reilly Media, 2018.
	\end{itemize}

\end{document}
