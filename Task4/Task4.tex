\documentclass[12pt]{article}
\usepackage{../common/cpp-lectures}

\title{Завдання 4. Клас Image. Частина 2}

\begin{document}
\maketitle
В цьому завданні ми будемо розширювати клас \m{Image}, дозволяючи йому працювати з різними форматами файлів за допомогою паттерну \href{https://refactoring.guru/uk/design-patterns/strategy/cpp/example}{\m{Strategy}}. Стратегія дозволяє нам мати екземпляр нашого класу, який може ``перемикатися'' між роботою в різних режимах, в нашому випадку, мова буде йти про різні формати зображень. Це завдання складається з двох під-завдань.

\begin{itemize}
  \item Інтерфейс стратегії.
    Підготуйте ваш існуючий клас \m{Image} для роботи в режимі стратегії:
    \begin{itemize}
      \item Створіть інтерфейс (чистий абстрактний клас) з обов'язковими спільними методами для всіх конкретних стратегій.
      \item Створіть клас, який втілює цю стратегію і перенесіть туди функції, написані раніше, для зчитування та запису PGMA файлів.
      \item Інтегруйте цю стратегію в клас \m{Image} через вказівник на інтерфейс.
    \end{itemize}

  \item Додаткові стратегії.
  Звичайно, використання цього паттерну має сенс за наявності декількох взаємозамінних стратегій. Створіть другу стратегію, для роботи з кольоровими файлами в PPM форматі. Довідку про PPM можна знайти, зокрема, \href{https://people.sc.fsu.edu/~jburkardt/data/ppma/ppma.html}{тут}. За великим рахунком, від PGMA він відрізняється тільки тим, що замість одного числа, що характеризує відтінок монохромного пікселя, їх там три -- для червоного, зеленого та синього, відповідно.\footnote{Цю трійку має сенс зберігати разом у вкладеній структурі \m{Image::Pixel}.} Цей новий клас має мати той самий функціонал, як і старий, що диктується інтерфейсом. Вибір стратегії може здійснюватись вручну за допомогою setter-методів, або автоматично зважаючи на розширення файлу на моменті створення екземпляру \m{Image} (виклик конструктора), або використання одного з його методів (зокрема зчитування).

\end{itemize}

\newpage
\begin{center}
    \large{Додатковий бал}
\end{center}

\begin{itemize}
\item В окремому файлі, спробуйте переписати клас \m{Image}, так, щоб він використовував поліморфізм напряму, замість стратегії. Тобто, мав похідні класи \m{PGMAImage} та \m{PPMImage}. В чому різниця між цими підходами коли один може буде кращим за інший? 
\item Напишіть автоматизовані тести, для тестування абстрактної стратегії, які зможуть працювати з будь-якою конкретною стратегією.
\item* Імплементуйте ще одну стратегію, яка дозволить нам працювати з PNG файлами. Для цього не варто намагатись розробити своє рішення. Натомість, треба використовувати наявні поширені та відтестовані бібліотеки. А саме \href{http://www.libpng.org/pub/png/libpng.html}{\m{libpng}}\footnote{Зауважте, в неї є залежність, інша бібліотека \m{zlib}.}, та її обгорткою на C++ \href{https://www.nongnu.org/pngpp/doc/0.2.9/}{\m{png++}}. В середі Ubuntu Linux файли для роботи з PNG зображеннями знаходяться в пакеті \m{libpng++}, що можна встановити наступним чином:
\begin{lstlisting}[language=sh]
sudo apt install libpng++-dev
\end{lstlisting}
Встановлення залежностей та підключення їх через CMake залишається частиною і важливим педагогічним аспектом цього завдання.

\end{itemize}


\end{document}