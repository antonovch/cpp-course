
\documentclass[12pt]{article}
\usepackage[utf8]{inputenc}
\usepackage[T2A]{fontenc}
\usepackage[english, ukrainian]{babel}
\usepackage[a4paper, total={6in, 8in}]{geometry}
\usepackage{amsmath}
\usepackage{graphicx}
\usepackage{hyperref}
\usepackage[shortlabels]{enumitem}

\title{Лекція 1. Вступ.}
\author{}
\date{Версія \today}

\begin{document}
    %\begin{titlepage}
	\maketitle
	%\end{titlepage}
    \abstract{Ця лекція охоплює знайомство з мовою програмування C++, огляд її основних властивостей, мотивація, налаштування середовища.}
	
	%\newpage
	\tableofcontents
	%\newpage
	
	\section{Чому слід вивчати та використовувати C++?}

	Мова програмування C++ має велику історію, широку стандартну бібліотеку та підтримку, велику спільноту розробників, дослідників та різного штибу прихильників, які використовують її у своєму професійному житті та/або для власних проектів.

	В наукових, інженерних або дослідницьких колах вона часто допомагає писати більш універсальний, а головне швидкий та ефективний код, коли простіші інструменти на кшталт MatLab або Python не підходять. Такі програми можуть зазвичай виконуватись на більшій кількості платформ.

	Сучасні стандарти (C++17 та C++20) роблять з неї у багатьох сенсах високорівневу мову програмування, яка тим не менш залишає розробнику можливість оптимізувати свої програми на більш ``низькому'' рівні, і яка максимально зберігає свою славетну швидкість та ефективність.

	Більшості, C++ відома як об'єктно-орієнтована мова програмування, але вона прекрасно годиться для інших парадигм програмування, або їх комбінації. С++, особливо останні стандарти, дозволяє розробнику (автору) чітко та неприховано висловлювати свої наміри при написанні коду, що майже робить непотрібними взагалі.\footnote{Дійсно, більшість програмістів вважають їх недоречними в сучасних програмах, окрім дуже окремих випадків.}
	Це досягається побудовою та використанням доречних структур та класів для своєї задачі, назвами, що допомагають зрозуміти роль та основні властивості того, що вони охрестили, та написанням продуманих абстракцій.

	\section{Що потрібно для того, щоб почати працювати з C++?}
	Щоб мати змогу отримувати файл, який можна запустити, з вашого вихідного коду, треба мати встановленим C/C++ компілятор. Для операційних систем Windows популярним вибором є Visual Studio C/C++ Compiler, для GNU/Linux та MacOS -- GCC від GNU або Clang від LLVM. В зв'язці з окремо взятим компілятором, як правило, їдуть набір супроводжуючих інструментів для форматування, аналізу, налагодження (debugging), заміру різних характеристик вихідного коду та скомпільованих програм. Радимо ознайомитись з можливостями того комплекту інструментів, який ви збираєтесь встановлювати.
	
	Для безпосереднього написання та налагодження роботи програми існує два підходи:
	\begin{enumerate}[(а)]
		\item використання інтегрованих середовищ розробки (IDE), наприклад, Visual Studio, Eclipse, CLion та ін.,
		\item використання сучасних гнучких текстових редакторів.
	\end{enumerate}

	Роблячи вибір для себе, слід зауважити наступне
	\begin{itemize}
		\item IDE це великі програми, які роблять багато чого. Часом надто багато. Вони можуть приховувати деякі технічні деталі того, що відбувається, що само по собі призвано привносити зручності і простоту в процес роботи. Проте, справжній спеціаліст мусить розуміти все, що відбувається за лаштунками, тому тим, хто що навчається, не варто пропускати якісь кроки або деталі.
		\item Знову ж таки, IDE це великі програми, вони намагаються максимально спростити роботу для вас, і тому вони насичені різноманітним функціоналом, якій не дуже може бути потрібним чи корисним, але на практиці це виливається в більшу витрату ресурсів комп'ютера при роботі, повільну роботу з текстом.
		\item Цілі IDE або деякі їх функції можуть бути платними.
		\item На противагу ним, сучасні текстові редактори на кшталт Visual Studio Code, Sublime Text та деякі інші, не відстають від найпросунутіших IDE в тому що стосується аналізу кода, навігації, автоматизації задач, статичному аналізі вихідного коду.
		\item Вони у деякому сенсі більш гнучкі -- деякі функції можуть не бути там за замовченням, \footnote{Велика вірогідність, що для вашого випадку вже існує безкоштовне розширення!} але вони можуть буди задані через системні команди, прописані в конфігураційних файлах. Це більш гнучкий, універсальний і природний для професійних програмістів підхід, ніж вишукування налаштування в графічному інтерфейсі окремо взятої IDE.
		\item Робота з ними більш плавна та вони забирають менше системних ресурсів.
		\item Для популярних редакторів, таких як Visual Studio Code, існує безліч безкоштовних додатків, які розширюють його можливості до рівня повноцінного IDE.
	\end{itemize}

	IDE, саме, часто просто інтегрують out-of-the-box інтеграцію з загальнодоступними інструментами, такими як
	\begin{itemize}
		\item \texttt{clang\_format} для форматування вихідного коду
		\item \texttt{cpplint}, \texttt{cppcheck}, \texttt{Clang-Tidy} і \href{https://en.wikipedia.org/wiki/List_of_tools_for_static_code_analysis#C,_C++}{багато інших} для аналізу коду до моменту компіляції.
	\end{itemize}

	Наостанок, зазначимо важливість вибору і дотримання стилю написання та оформлення коду. Це стосується відступів, назв, розміру функцій перед тим як їх варто розділити на під-функції, максимальна ширина строки, положення дужок, заборона на використання потенційно ``небезпечних'' можливостей, переваги при виборі типів і багато-багато іншого. Непоганою вихідною точною є \href{https://google.github.io/styleguide/cppguide.html}{Google Code Style Sheet}. Також корисним буде ознайомитись з \href{https://github.com/isocpp/CppCoreGuidelines}{C++ Core Guidelines}.

	\section*{Завдання}
	\addcontentsline{toc}{section}{Завдання}
	
	Налаштувати свій робочий комп'ютер під роботу з C++: встановити все необхідне програмне забезпечення, скомпілювати програму ``Hello world'' та інші доступні на своєму комп'ютері, переконатись в коректній роботі.

\end{document}
