\documentclass[12pt]{article}
\usepackage{../common/cpp-lectures}

\title{Завдання 2. Динамічна генерація рядків}

\begin{document}
\maketitle

В цьому завданні ми напишемо невеличку бібліотеку для конвертування інформації, що зберігається в \m{std::map} у строку для подальшого запису в файл в одному чи декількох форматах: XML, JSON, YAML. Ця процедура відома як серіалізація. Синтаксис кожного формату можна знайти в інтернеті, але для довідки, можна користуватись відповідним \m{example.*} файлом в цій директорії.
Ключ (\m{key\_type}) для нашої \m{std::map} мусить бути об'єктом типу \m{std::string}. Значення може бути одне з трьох:
\begin{enumerate}
  \item аналогічний об'єкт \m{std::map},
  \item об'єкт \m{std::vector<std::string>},
  \item об'єкт \m{std::string}.
\end{enumerate}

Файл заголовка \m{serializer.h} має в собі всі декларації, які необхідні для роботи. Для виконання завдання необхідно імплементувати по три функції для кожного формату для кожного з трьох типів даних в файлі \m{serializer.cpp}. Шаблоні функції, задані в \m{serializer.h} викличуть необхідну функцію для кожного типу даних.

Файл \m{main.cpp} дає приклад використання цієї бібліотеки. Його можна використовувати як перший етап верифікації роботи програми при її розробці -- вона повинна писати текст, аналогічний тому, що знаходиться в файлах \m{example.*}.\footnote{Зауважте, що порядок виводу елементів на екран, тобто порядок їх додавання у вихідну строку, відрізняється від порядку їх додавання в \m{std::map}, який співпадає з тим, як вони записані в \m{example.*}. Це через те, що \m{std::map} сортує свої пари використовуючи оператор \m{<} своїх ключів.}

\begin{center}
    \large{Додатковий бал}
\end{center}

\begin{itemize}
\item Зробіть імплементацію для всіх трьох форматів даних.
\item Запишіть кожен результат у файл зі своїм розширенням.
\item Напишіть будь-які функції для автоматизованого тестування. Тобто, такі, що передають заздалегідь визначений об'єкт типу \m{std::map} з різною глибиною вкладення у серіалізатор, отримують рядок, і перевіряють з еталоном, який відповідає переданим даним.
\end{itemize}

\end{document}