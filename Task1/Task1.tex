\documentclass[12pt]{article}
\usepackage[utf8]{inputenc}
\usepackage[T2A]{fontenc}
\usepackage[english, ukrainian]{babel}
\usepackage[a4paper, total={6in, 8in}]{geometry}
\usepackage{amsmath}
\usepackage{graphicx}
\usepackage{hyperref}
\usepackage{listings}
\usepackage{color}

\definecolor{dkgreen}{rgb}{0,0.6,0}
\definecolor{gray}{rgb}{0.5,0.5,0.5}
\definecolor{mauve}{rgb}{0.58,0,0.82}

\lstset{frame=tb,
  language=sh,
  aboveskip=3mm,
  belowskip=3mm,
  showstringspaces=false,
  columns=flexible,
  basicstyle={\small\ttfamily},
  numbers=none,
  numberstyle=\tiny\color{gray},
  keywordstyle=\color{blue},
  commentstyle=\color{dkgreen},
  stringstyle=\color{mauve},
  breaklines=true,
  breakatwhitespace=true,
  tabsize=3
}

\title{Завдання 1. Обробка вхідних аргументів програми.}
\date{}
\author{}

\begin{document}
\maketitle

Ціль цього завдання — написати просту програму для практики важливих базових технік доступних користувачу C++. Програма має робити просту річ: зчитувати список переданих їй аргументів, які в нашому випадку будуть назвами уявних (неіснуючих) файлів з їх розширенням (але без шляху чи директорії), і казати користувачу скільки файлів якого типа вона нарахувала. 

Наприклад,

\begin{lstlisting}
> ./build/task1 myfile.txt mydoc.doc yourdoc.doc otherfile.txt
You have provided 2 txt and 2 doc files.
\end{lstlisting}

Формат вихідного повідомлення залишається на вибір автора, але має бути зрозумілим, інформативним і граматично вірним.

\begin{center}
    \large{Додатковий бал}
\end{center}

\begin{itemize}
\item щоб трохи розширити свою програму, збережіть назви файлів у stringsteam, std::vector, або інше за вибором, і покажіть користувачу списки файлів для кожного розширення окремо,
\item напишіть найпростіші автоматизовані тести для вашої програми.\footnote{Це незамінна навичка, якщо ви плануєте так чи інакше пов'язати ваше життя з розробкою чи тестуванням програм.}
Для цього можна зробити наступне
\begin{enumerate}
    \item Перенесіть основну логіку з main() в окрему функцію, щоб її можна було перевіряти.
    \item Створіть функцію-тестувальницю.
    \item Придумайте декілька різних типових (або навпаки неочікуваних!) варіантів вводу від користувача.
    \item Всередині, оформіть їх в якусь форму, що буде зрозумілою для функції, що ви тестуєте.
    \item Там же, викличте функцію-рахувальницю з різними варіантами списків файлів, перевірте коректність виводу.\footnote{Якщо ви не хочете прив'язуватись до конкретного тексту повідомлення, перевіряйте тільки те, що має значення на вашу думку. Наприклад, у виводі програми, що наведено вище, достатньо пересвідчитись, що повідомлення має наступні підрядки ``2 txt'' та ``2 doc''.}
    \item У main(), перед тим як перевіряти ввод користувача, викличте тести та виведіть одне з двох повідомлень: ``All tests passed!'', або ж  ``Some tests failed!''.
\end{enumerate}
\end{itemize}

\end{document}